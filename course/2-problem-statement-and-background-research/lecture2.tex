\documentclass[aspectratio=169]{beamer}

\newcommand{\myroot}{../..}
\usepackage[slides]{\myroot/course}
\title{\usnaCourseNumber\ Guided Design Experience}
\subtitle{Problem statement and background research, \usnaCourseTerm}
\author{\usnaInstructorShort}
\date{\today}

\begin{document}
\frame{\titlepage}

\frame{\frametitle{Defining the problem}
\begin{columns}
\begin{column}{0.66\textwidth}
\begin{itemize}
\item Problem statement introduction
\item Developing a good problem statement
\item Role of background research
\item Assignment \#1
\end{itemize}
\end{column}
\begin{column}{0.33\textwidth}
\begin{center}
includegraphic here
\end{center}
\end{column}
\end{columns}
}

\frame{\frametitle{Qualities of a good problem statement}
\begin{itemize}
\item Captures the full scope of the effort
\begin{itemize}
\item What the design \emph{will} and \emph{will not} do
\end{itemize}
\item Still leaves room for design
\begin{itemize}
\item Loosely structured (to allow flexibility insolution and adherence to constraints)
\item Open-ended for multiple solution sets
\end{itemize}
\item Avoids:
\begin{itemize}
\item Implicit or preconceived solutions
\item Biases
\item Assumptions
\end{itemize}
\end{itemize}
}

\frame{\frametitle{Can you identify the issues with this problem statement?}
\begin{itemize}
\item Client's original problem statement:
\begin{quotation}
An ecologically concerned client has asked you to help them change their home heating system from electricity to gas in order to reduce their heating cost.
\end{quotation}

\item Suggest a refined problem statement:
\begin{quotation}
Identify the most cost-effective and ecologically friendly change the client can make to reduce their home energy heating cost.
\end{quotation}
\end{itemize}
} % animated GIF adds what? 

\frame{\frametitle{How to develop a problem statement}
Given a preliminary design goal or challenge, what should you do?
\begin{enumerate}
\item Ask questions -- lots of them
\begin{enumerate}
\item What are my system requirements?
\item What are my system constraints?
\end{enumerate}
\item Research different designs and gather data
\begin{enumerate}
\item What has been done already?
\item What worked and what didn't work?
\end{enumerate}
\item Identify any artificial limitations or biases I may already have towards a specific solution
\item Draft an initial problem statement
\end{enumerate}
} % Dilbert comic adds what? 

\frame{\frametitle{The role of background research}
\begin{itemize}
\item Background research provides context for framing a good problem statement
\begin{itemize}
\item It gives you insight on the project requirements and constraints
\end{itemize}
\item It also allows you to investigate what others have done for similar problems and the challenges they faced
\begin{itemize}
\item What approaches worked?
\item What approaches didn't work?
\item How can the design be enhanced? 
\end{itemize}
\end{itemize}
} % image of Dory adds nothing? 

% This is kind of absurd! The original slide has 5 levels of 
% hierarchical bullets; Latex won't even let me do that. 
\frame{\frametitle{Assignment \#1}
\begin{itemize}
\item As a project team, develop a draft problem statement for your autonomous turret system
\begin{itemize}
\item Follow the guidance provided in these slides and the class discussions
\end{itemize}
\item As an individual, research three different types of automated turret designs
\begin{itemize}
\item Can be fully operational, Kickstarter projects, or even capstone-like projects
\item For each of the designs, must provide
\begin{itemize}
\item Functions (the how) - how does it sense, decide, and actuate?
%\begin{itemize}
%\item How does it sense, decide, and actuate?
%\end{itemize}
\item Objectives (the what) - is it portable, scalable, robust, cheap? 
%\begin{itemize}
%\item Is it portable? Scalable? Robust? Cheap? 
%\end{itemize}
\item Constraints (the don't even go there) - safety, size, cost, feasibility
%\begin{itemize}
%\item Safety, size, cost, feasibility
%\end{itemize}
\end{itemize}
\end{itemize}
\end{itemize}
}

\end{document}
