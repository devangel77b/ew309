\documentclass[aspectratio=169]{beamer}
\usepackage[utf8]{inputenc}
\usepackage[scaled]{helvet}
\renewcommand\familydefault{\sfdefault}
\usepackage[T1]{fontenc}

\newcommand{\myroot}{../..}
\usepackage[slides]{\myroot/course}
\title{Building the firing circuit}
\subtitle{\usnaCourseNumber\ Guided Design Experience, \usnaCourseTerm}
\author{\usnaInstructorShort}
\date{\today}
	
\usetheme{Hopper}

\begin{document}
\settitlebg
\begin{frame}
\titlepage
\end{frame}

\setslidebg
\begin{frame}
\frametitle{The what and why of it}
In EW202, you used the mbed processor as a decision maker. So can we do this tom ake our Nerf motors spin when we want?
\begin{center}
graphics here
\end{center}
Just use a \lstinline{DigitalOut} line to turn on our motors, right? 
\end{frame}

\begin{frame}
\frametitle{Let's think about it another way}
What if $R_a=\SI{5}{\ohm}$? Replacing the DC motor with its schematic, at stall $\omega=0$, $i_a=\SI{0.66}{\ampere}$. 
\end{frame}

\begin{frame}
\frametitle{From the mbed datasheet}
\begin{itemize}
\item Current limited to \SI{500}{\milli\ampere}
\item Digital IO pins are \SI{3.3}{\volt}, \SI{4}{\milli\ampere} each; \SI{400}{\milli\ampere} total
\end{itemize}
\end{frame}

\begin{frame}
\frametitle{So we need something else...}
\end{frame}

\begin{frame}
\frametitle{What is this ``something''?}
\begin{itemize}
\item What are its \emph{functions}, i.e. what does it need to do?
\item This ``something'' must be able to do the following (functional requirements):
\begin{itemize}
\item Controlled by a digital IO line (\SIrange{0}{3.3}{\volt})
\item Draw no more than \si{\milli\ampere} on the mbed side
\item Supply on the order of \si{\ampere} on the motor side
\end{itemize}
\item Now, what options do we have? 
\end{itemize}
\end{frame}

\begin{frame}
\frametitle{Option 1: Relay (electromechanical)}
A relay is an electrically operated switch that uses an electromagnet to mechnically throw a switch to open and close circuits. 
\end{frame}

\begin{frame}
\frametitle{Option 2: Metal oxide semiconductor field-effect transistor ({MOSFET})}
A MOSFET is a type of transistor that can be used as a \emph{solid state} switch, to switch electronic signals. 
\end{frame}

\begin{frame}
\frametitle{Relay vs {MOSFET}}
\begin{columns}
\begin{column}{0.5\textwidth}
Electromechnical relay
\begin{itemize}
\item Relatively slow switching frequency (too slow for speed control or dimmable lights)
\item Noisy when switching
\item Lifetime number of on/off cycles is limited by mechanical wear
\item \emph{Seemingly} simple to implement; it's basically an electronic switch
\item Commonly used in telecommunications, can be used to switch AC line voltages
\end{itemize}
\end{column}
\begin{column}{0.5\textwidth}
$n$-channel MOSFET
\begin{itemize}
\item High switching frequency, can implement PWM from mbed
\item Low noise
\item Scary data sheet (maybe?), must utilize additional circuitry?
\item Commonly used in DC applications like motor control
\end{itemize}
\end{column}
\end{columns}
\end{frame}

\begin{frame}
\frametitle{Relay vs {MOSFET} implementation}
\begin{columns}
\begin{column}{0.5\textwidth}
Electromechanical relay
\end{column}
\begin{column}{0.5\textwidth}
$n$-channel MOSFET
\end{column}
\end{columns}
\end{frame}

\begin{frame}
\frametitle{Circuit implementation}
\end{frame}

\begin{frame}
% use sequential build here? 
\frametitle{Relay vs {MOSFET} implementation for testing}
\begin{columns}
\begin{column}{0.5\textwidth}
Electromechanical relay
\end{column}
\begin{column}{0.5\textwidth}
$n$-channel MOSFET
\end{column}
\end{columns}
\end{frame}

\begin{frame}
\frametitle{{MOSFET} circuit implementation}
\framesubtitle{Unidirectional motor driver}
Now let's discuss the circuit operation. The switch is shut and the MOSFET conncts the drain to source through a very small (i.e., short) resistance (\si{\milli\ohm}). 
\end{frame}

\begin{frame}
% Use sequential build here? 
\frametitle{{MOSFET} circuit implementation}
\framesubtitle{Unidirectional motor driver}
Now the switch is opened. What happens? 
\end{frame}

\begin{frame}
\frametitle{Your task...}
\begin{enumerate}
\item Select the circuit you want to drive your Nerf motors.
\item Look at the datasheet for your selection and verify that all components work with the voltage and current requirements for your motor.
\item Get a breadboard from your instructor.
\item Build \emph{two} (one for each motor) of the firing circuits on the protoboard using the manual switches and external power supply. \textbf{Do not connect your mbed just yet.}
\item Verify that you have established \textbf{a common ground}.
\item Be sure to document all your steps.  You will have write these up for your next assignment and in your final report.
\end{enumerate}
\end{frame}

\end{document}
