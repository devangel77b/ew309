\documentclass[aspectratio=169]{beamer}
\usepackage[utf8]{inputenc}
\usepackage[scaled]{helvet}
\renewcommand\familydefault{\sfdefault}
\usepackage[T1]{fontenc}

\newcommand{\myroot}{../..}
\usepackage[slides]{\myroot/course}
\title{Overview of snubber circuits}
\subtitle{\usnaCourseNumber\ Guided Design Experience, \usnaCourseTerm}
\author{J McGillick, USNA 2017}
\date{\today}
	
\usetheme{Hopper}

\begin{document}
\settitlebg
\begin{frame}
\titlepage
\end{frame}

\setslidebg
\begin{frame}
\frametitle{What are snubber circuits?}
\begin{columns}
\begin{column}{0.5\textwidth}
\begin{itemize}
\item Protect the switch from inductive element
\item Can consist of multiple materials
\item Used to direct current from inductor in safe direction
\end{itemize}
See also \href{https://www.youtube.com/watch?v=LXGtE3X2k7Y}{video explanation}
\end{column}
\begin{column}{0.5\textwidth}
\begin{center}
\includegraphics[width=\columnwidth]{\myroot/figures/l5-f2.png}
\end{center}
\end{column}
\end{columns}
\end{frame}

\begin{frame}
\frametitle{Why do we need one?}
\begin{columns}
\begin{column}{0.5\textwidth}
\begin{itemize}
\item DC motors consist of windings to make it work
\item Inductors oppose changes in current
\item DC motors will not allow for sudden change in current when shut off
\end{itemize}
\end{column}
\begin{column}{0.5\textwidth}
\begin{center}
\includegraphics[width=\columnwidth]{\myroot/figures/l5-f3.png}
\end{center}
\end{column}
\end{columns}
\end{frame}

\begin{frame}
\frametitle{What are we using?}
\begin{itemize}
\item Diode allows current to flow through it
\item Resistor dissipates power
\item Motor does not burn out MOSFET
\end{itemize}
\begin{center}
\includegraphics[width=0.5\columnwidth]{\myroot/figures/l5-f4.png}
\end{center}
\end{frame}

\end{document}
