\documentclass[noanswers]{exam}

\newcommand{\myroot}{../../..}
\usepackage[hw]{\myroot/course}
\title{Assignment \#3: Functional block diagram(s)}
\author{\usnaAuthorShort}
%\date{\printdate{\hwZeroOut}}
%\duedate{\printdate{\hwzeroDue}}
\date{\today}
\duedate{\printdate{2/12/2020}}

\begin{document}
\maketitle

For all assignments, upload a copy of the assignment as a \Google\ document to \GoogleClassroom\ and rename it with the assignment name and team member’s last name.  For example:
\begin{lstlisting}
3-functional-block-diagram-evangelista-binzel
\end{lstlisting}



\begin{questions}
\question
Start working on a functional block diagram(s) for your auto-turret (we will add to this as the semester goes on). Begin by asking:
\begin{itemize}
\item What subsystems does your project currently have?
\item What components belong to each of these subsystems?
\item For each subsystem and component, ask
\begin{itemize}
\item What are the inputs?
\item What are the outputs?
\item What type of connections are required
\end{itemize}
\item You may choose to draw a single diagram, or many; the choice will depend on what you wish to communicate with each diagram
\end{itemize}

\question 
Represent each subsystem and necessary components as blocks in your diagram(s).%\footnote{Note that not all components need to be separate blocks but each subsystem should be block on your functional block diagram}

\question
Make connections between your subsystems and components.  Be sure to label each connection. You may choose to label the type of connection (electrical, mechanical, data, other), and/or the type of information or data being shared (\SI{9}{\volt} power, serial commands, PWM signal). Use of color or a legend is helpful as it can visually inform a reader of how each type of connection is represented on your diagram(s). 

\question
\textbf{Statement of contribution.} Since a majority of the work in the class is done as a team, it is also important to identify the individual contributions of team member.  For each assignment, include a description and percentage of each member’s contribution.
\end{questions}
\end{document}

