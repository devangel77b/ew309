\documentclass[noanswers]{exam}

\newcommand{\myroot}{../../..}
\usepackage[hw]{\myroot/course}
\title{Assignment \#10: Post system integration}
\author{\usnaAuthorShort}
%\date{\printdate{\hwZeroOut}}
%\duedate{\printdate{\hwzeroDue}}
\date{\today}
\duedate{\printdate{4/27/2020}}

\begin{document}
\maketitle

This assignment should be included as a section in your final report in Overleaf.  You can share the document with me or just upload a pdf copy of your Overleaf document thus far in Google Classroom. 
You only need to submit one assignment per group but each member should contribute equally to the completion of the steps.  

\begin{questions}
\question \textbf{So what?} You have completed system integration of the computer vision, closed loop turret position control, and ballstics subsystems using a virtual turret system. What did you learn? The final (\textbf{``Discussion''}) segment of your report should reflect on what you as an engineer would take away from the effort moving forward.

\begin{parts}
\part \textbf{Facts.} Briefly review what you found, and the significance of your findings. 
\part \textbf{Discussion.} From having simulated the integrated turret system, what did you learn?
\begin{subparts}
\subpart Did things work as expected? Did anything surprise you? 
\subpart Have any major risks been retired? Have any new risks been identified for moving to physical hardware? Based on everything you've learned so far, how much effort would it take to jump to system integration with the physical hardware? 
\subpart Are there any new actions you would take based on what you learned with the simulation? 
\end{subparts}
\part \textbf{Action.} Overall, what would you recommend as the next step in nerf gun autoturret development, balancing your understanding of customer needs, cost, schedule, risk, and technical challenges. 
\end{parts}


\end{questions}
\end{document}


