\documentclass[noanswers]{exam}

\newcommand{\myroot}{../../..}
\usepackage[hw]{\myroot/course}
\title{Assignment \#9: Test plan and ballistics analysis}
\author{\usnaAuthorShort}
%\date{\printdate{\hwZeroOut}}
%\duedate{\printdate{\hwzeroDue}}
\date{\today}
\duedate{\printdate{4/23/2020}}

\begin{document}
\maketitle

This assignment should be included as a section in your final report in Overleaf.  You can share the document with me or just upload a pdf copy of your Overleaf document thus far in Google Classroom. 
You only need to submit one assignment per group but each member should contribute equally to the completion of the steps.  

\begin{questions}
\question \textbf{Test plan and procedures.} Develop a detailed test plan on how you would collect data on the accuracy and precision of your weapon if you had access to the \usnaCourseNumber\ hardware.  Your test plan should be sufficient for someone else to replicate your testing. The test plan may include the follow items (note this is not an exhaustive list)
\begin{parts}
\part How would you know where your weapon is aiming?
\part How would you know where your shot landed?
\part What tools would you need to measure the precision and accuracy?
\part How many shots would you need to take to get a good representative grouping?
\part How many different ranges would you need to shoot from?
\end{parts}



\question \textbf{Shot pattern analysis.}
\begin{parts}
\part Using the shot data provided by your instructor from four different distances, compute the following statistics for each range
\begin{subparts}
\subpart $x$ bias (mean)
\subpart $y$ bias (mean)
\subpart Precision error ($S_p$)
\subpart Circular error probable (CEP) (for 50\%)
\end{subparts}    
\part Construct a table similar to the one below for your data
\begin{table}[ht]
\begin{center}
\begin{tabular}{ccccc}
\toprule
range (\si{\centi\meter}) & $x$ bias (\si{\centi\meter}) & $y$ bias (\si{\centi\meter}) & precision error (\si{\centi\meter}) & number of shots \\
\midrule
x & x & x & x & x \\
\bottomrule
\end{tabular}
\end{center}
\end{table}
\part\label{part:c} Plot your precision error ($S_p$) vs distance (range) and fit an appropriate degree polynomial to the data.  Since CEP is a function of precision error ($S_p$) and the scale factor ($k$), we can use the same fit to calculate our CEPs vs. range for any scale factor. Comment on the fit; does it match the data well?  Why or why not?
\part Repeat step~\ref{part:c} for the $x$ and $y$ bias data respectively and find an appropriate fit for each dimension.  Comment on the fit; does it match the data well?  Why or why not?
\part Compute the 50\% CEP for each range as well as the 95\% confidence intervals on each CEP.  Note this is the same as finding a confidence interval about your precision error ($S_p$) at each range.
\part Generate scatter plots (Point of Aim (POA) and Points of Impact (POI)) as well as your 50\% CEP and 95\% confidence intervals for your bias corrected data.  Be sure to label your axes correctly! Remember that all figures must be referenced appropriately in your Overleaf document.
\part Write a function that takes range, target size, and probability for one hit as input arguments and returns the $x$ and $y$ bias, angle correction in both dimensions ($x$ and $y$) as well as a number of shots for the probability of at least one hit.
\end{parts}



\question Include all the information above in the ballistics analysis subsection of your report.  Ensure that the section includes:
\begin{parts}
\part Complete sentences with no spelling errors.
\part Test plan and procedures
\part Table of ballistic statistical data for all four ranges
\part Scatter plots for each distance (with labeled axes) including POI, CEP and 95\% confidence intervals. Don’t forget to correct for bias first!
\part The polynomial fit and plots for the $x$ bias, $y$ bias, and precision error ($S_p$)
\part The code for the function generated including documentation
\end{parts}
\end{questions}
\end{document}


