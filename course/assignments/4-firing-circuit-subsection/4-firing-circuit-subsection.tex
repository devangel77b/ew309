\documentclass[noanswers]{exam}

\newcommand{\myroot}{../../..}
\usepackage[hw]{\myroot/course}
\title{Assignment \#4: Firing circuit}
\author{\usnaAuthorShort}
%\date{\printdate{\hwZeroOut}}
%\duedate{\printdate{\hwzeroDue}}
\date{\today}
\duedate{\printdate{2/20/2020}}

\begin{document}
\maketitle

For all assignments, upload a copy of the assignment as a \Google\ document to \GoogleClassroom\ and rename it with the assignment name and team member’s last name.  For example:
\begin{lstlisting}
4-firing-circuit-evangelista-binzel
\end{lstlisting}



\begin{questions}
\question
\textbf{Firing Circuit Subsection.} In your draft report, add a subsection describing the firing circuit. 
\begin{itemize}
\item Introduce the firing circuit and explain why it is needed
\item Provide a circuit schematic and a photo of your actual circuit with captions and callouts to the main parts of the circuit.
\item Discuss the following questions in your write-up
\begin{itemize}
\item What is the role of the pull-down resistor?
\item What is the purpose of the MOSFET or relay (depending on which one you implemented)
\item What is the purpose of a flyback diode and snubber circuit (the diode and resistor across the drain and source pins)?
\end{itemize}
\item Discuss any challenges you had in building the firing circuit.  How did you handle the challenges?
\end{itemize}

A reader should be able to duplicate your results based on your presentation.

This section (and all follow-on sections) must also include a critical analysis of your procedure, implementation, and results for full credit.  If you had any issues, explain your troubleshooting process and any lessons learned.

\question
\textbf{Statement of contribution.} Since a majority of the work in the class is done as a team, it is also important to identify the individual contributions of team member.  For each assignment, include a description and percentage of each team member’s contribution. 
\end{questions}
\end{document}

