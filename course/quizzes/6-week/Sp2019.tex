%%%%%%%%%%%%%%%%%%%%%%%%%%%%%%%%%%%%%%%%%
% Structured General Purpose Assignment
% LaTeX Template
%
% This template has been downloaded from:
% http://www.latextemplates.com
%
% Original author:
% Ted Pavlic (http://www.tedpavlic.com)
%
% Note:
% The \lipsum[#] commands throughout this template generate dummy text
% to fill the template out. These commands should all be removed when 
% writing assignment content.
%
%%%%%%%%%%%%%%%%%%%%%%%%%%%%%%%%%%%%%%%%%

%----------------------------------------------------------------------------------------
%	PACKAGES AND OTHER DOCUMENT CONFIGURATIONS
%----------------------------------------------------------------------------------------

\documentclass{article}

\usepackage{fancyhdr} % Required for custom headers
\usepackage{lastpage} % Required to determine the last page for the footer
\usepackage{extramarks} % Required for headers and footers
\usepackage{graphics}	% for pdf, bitmapped graphics files
\usepackage{graphicx} % Required to insert images
\usepackage{float} % enables [H] to force figure placement
\usepackage{amsmath}	% assumes amsmath package installed
\usepackage{amssymb}	% assumes amsmath package installed
\usepackage{lipsum} % Used for inserting dummy 'Lorem ipsum' text into the template
\usepackage{enumerate}
%\usepackage{mcode}

% Margins
\topmargin=-0.45in
\evensidemargin=0in
\oddsidemargin=0in
\textwidth=6.5in
\textheight=9.0in
\headsep=0.25in 

\linespread{1.1} % Line spacing

% Set up the header and footer
\pagestyle{fancy}
\lhead{\hmwkAuthorName} % Top left header
%\chead{\hmwkClass\ (\hmwkClassInstructor\ \hmwkClassTime): \hmwkTitle} % Top center header
\rhead{\hmwkClass\ \hmwkTitle, Due: \hmwkDueDate} % Top center header
%\rhead{\firstxmark} % Top right header
\lfoot{\lastxmark} % Bottom left footer
\cfoot{} % Bottom center footer
\rfoot{Page\ \thepage\ of\ \pageref{LastPage}} % Bottom right footer
\renewcommand\headrulewidth{0.4pt} % Size of the header rule
\renewcommand\footrulewidth{0.4pt} % Size of the footer rule

\setlength\parindent{0pt} % Removes all indentation from paragraphs

%----------------------------------------------------------------------------------------
%	DOCUMENT STRUCTURE COMMANDS
%	Skip this unless you know what you're doing
%----------------------------------------------------------------------------------------

% Header and footer for when a page split occurs within a problem environment
\newcommand{\enterProblemHeader}[1]{
\nobreak\extramarks{#1}{#1 continued on next page\ldots}\nobreak
\nobreak\extramarks{#1 (continued)}{#1 continued on next page\ldots}\nobreak
}

% Header and footer for when a page split occurs between problem environments
\newcommand{\exitProblemHeader}[1]{
\nobreak\extramarks{#1 (continued)}{#1 continued on next page\ldots}\nobreak
\nobreak\extramarks{#1}{}\nobreak
}

\setcounter{secnumdepth}{0} % Removes default section numbers
\newcounter{homeworkProblemCounter} % Creates a counter to keep track of the number of problems

\newcommand{\homeworkProblemName}{}
\newenvironment{homeworkProblem}[1][Problem \arabic{homeworkProblemCounter}]{ % Makes a new environment called homeworkProblem which takes 1 argument (custom name) but the default is "Problem #"
\stepcounter{homeworkProblemCounter} % Increase counter for number of problems
\renewcommand{\homeworkProblemName}{#1} % Assign \homeworkProblemName the name of the problem
\section{\homeworkProblemName} % Make a section in the document with the custom problem count
\enterProblemHeader{\homeworkProblemName} % Header and footer within the environment
}{
\exitProblemHeader{\homeworkProblemName} % Header and footer after the environment
}

\newcommand{\problemAnswer}[1]{ % Defines the problem answer command with the content as the only argument
\noindent\framebox[\columnwidth][c]{\begin{minipage}{0.98\columnwidth}#1\end{minipage}} % Makes the box around the problem answer and puts the content inside
}

\newcommand{\homeworkSectionName}{}
\newenvironment{homeworkSection}[1]{ % New environment for sections within homework problems, takes 1 argument - the name of the section
\renewcommand{\homeworkSectionName}{#1} % Assign \homeworkSectionName to the name of the section from the environment argument
\subsection{\homeworkSectionName} % Make a subsection with the custom name of the subsection
\enterProblemHeader{\homeworkProblemName\ [\homeworkSectionName]} % Header and footer within the environment
}{
\enterProblemHeader{\homeworkProblemName} % Header and footer after the environment
}
   
%----------------------------------------------------------------------------------------
%	NAME AND CLASS SECTION
%----------------------------------------------------------------------------------------

\newcommand{\hmwkTitle}{6-Week\ Quiz} % Assignment title
\newcommand{\hmwkDueDate}{February\ 13,\ 2020} % Due date
%\newcommand{\hmwkClass}{BIO\ 101} % Course/class
\newcommand{\hmwkClass}{EW309} % Course/class
%\newcommand{\hmwkClassTime}{10:30am} % Class/lecture time
\newcommand{\hmwkClassInstructor}{Severson} % Teacher/lecturer
\newcommand{\hmwkAuthorName}{Name:} % Your name

%----------------------------------------------------------------------------------------
%	TITLE PAGE
%----------------------------------------------------------------------------------------
%
%\title{
%\vspace{2in}
%\textmd{\textbf{\hmwkClass:\ \hmwkTitle}}\\
%\normalsize\vspace{0.1in}\small{Due\ on\ \hmwkDueDate}\\
%\vspace{0.1in}\large{\textit{\hmwkClassInstructor\ \hmwkClassTime}}
%\vspace{3in}
%}
%
%\author{\textbf{\hmwkAuthorName}}
%\date{} % Insert date here if you want it to appear below your name
%
%----------------------------------------------------------------------------------------

\begin{document}

%\maketitle

%----------------------------------------------------------------------------------------
%	TABLE OF CONTENTS
%----------------------------------------------------------------------------------------

%\setcounter{tocdepth}{1} % Uncomment this line if you don't want subsections listed in the ToC

%\newpage
%\tableofcontents
%\newpage

%----------------------------------------------------------------------------------------
%	PROBLEM 1
%----------------------------------------------------------------------------------------

% To have just one problem per page, simply put a \clearpage after each problem

%%%%%%%%%%%%%%%%%%%%%%%%%%%%%%%%%%%%%%%%%%%%%%%%%%%%%%%%%%%%%%%%%%%%%%%%%%%%%%%%%%
\begin{homeworkProblem}[Question 1] % \#\arabic{homeworkProblemCounter}]
A good problem statement for your automated turret system would accomplish the following (circle {\bf all} that apply):

\begin{enumerate}[(a)] % (a), (b), (c), ...
\item Allow for design trade space and open-ended solutions
\item Ensure the design looks exactly like the ones from the previous year  
\item Include any constraints or clarifying data that scope the problem
\item Dictate the type of control algorithm required at the start of the project
\item Clarify both what the design will and won't do 
\item Consider any external interactions and influences that could affect the design
\item Be very generic such that the objectives and design goals could be anything!
\end{enumerate}

%\vfill

\end{homeworkProblem}
%%%%%%%%%%%%%%%%%%%%%%%%%%%%%%%%%%%%%%%%%%%%%%%%%%%%%%%%%%%%%%%%%%%%%%%%%%%%%%%%%%

%%%%%%%%%%%%%%%%%%%%%%%%%%%%%%%%%%%%%%%%%%%%%%%%%%%%%%%%%%%%%%%%%%%%%%%%%%%%%%%%%%
\begin{homeworkProblem}[Question 2] % \#\arabic{homeworkProblemCounter}]
The purpose of presenting thorough background research in an engineering report or publication is to (circle {\bf all} that apply):

\begin{enumerate}[(a)] % (a), (b), (c), ...
\item Justify the contribution by comparing it to existing projects
\item Increase the length of the document 
\item Convince the reader that you are not ``reinventing the wheel'' (i.e. repeating an existing and fully documented project)
\item Demonstrate a understanding of the field
\item Background research serves no real purpose
\end{enumerate}

%\vfill

\end{homeworkProblem}
%%%%%%%%%%%%%%%%%%%%%%%%%%%%%%%%%%%%%%%%%%%%%%%%%%%%%%%%%%%%%%%%%%%%%%%%%%%%%%%%%%

%%%%%%%%%%%%%%%%%%%%%%%%%%%%%%%%%%%%%%%%%%%%%%%%%%%%%%%%%%%%%%%%%%%%%%%%%%%%%%%%%%
\begin{homeworkProblem}[Question 3] % \#\arabic{homeworkProblemCounter}]
What is the purpose of the component shown in Fig.~\ref{fig:REG3ASW} (circle {\bf one})?
\begin{figure}[H]
\centering
	\includegraphics[width=2.0in]{MOSFET.jpg}
	\caption{50N06 MOSFET}
	\label{fig:REG3ASW}
\end{figure}

\begin{enumerate}[(a)] % (a), (b), (c), ...
\item Allow bidirectional control of a motor using logic level PWM and a direction pin
\item Supply on the order of A to a motor while only drawing mA  
\item Look nice in your firing circuit
\item Prevent current from flowing back into an mBed
\end{enumerate}

\vfill

\end{homeworkProblem}
%%%%%%%%%%%%%%%%%%%%%%%%%%%%%%%%%%%%%%%%%%%%%%%%%%%%%%%%%%%%%%%%%%%%%%%%%%%%%%%%%%

%%%%%%%%%%%%%%%%%%%%%%%%%%%%%%%%%%%%%%%%%%%%%%%%%%%%%%%%%%%%%%%%%%%%%%%%%%%%%%%%%%
\begin{homeworkProblem}[Question 4] % \#\arabic{homeworkProblemCounter}]
What is the maximum output current rating for the component shown in Fig.~\ref{fig:REG3ASW} (circle {\bf one})?

\begin{enumerate}[(a)] % (a), (b), (c), ...
\item 500 mA
\item 1.0 A
\item 50.0 A
\item I have no clue how to read a datasheet
\end{enumerate}

\vfill

\end{homeworkProblem}
%%%%%%%%%%%%%%%%%%%%%%%%%%%%%%%%%%%%%%%%%%%%%%%%%%%%%%%%%%%%%%%%%%%%%%%%%%%%%%%%%%

%%%%%%%%%%%%%%%%%%%%%%%%%%%%%%%%%%%%%%%%%%%%%%%%%%%%%%%%%%%%%%%%%%%%%%%%%%%%%%%%%%
\begin{homeworkProblem}[Question 5] % \#\arabic{homeworkProblemCounter}]
What is the purpose of the components shown in Fig.~\ref{fig:Snubber} (circle {\bf one})?
\begin{figure}[H]
\centering
	\includegraphics[width=2.3in]{Snubber_Circuit.jpg}
	\caption{Manual Firing Circuit with important components highlighted and circled in red.}
	\label{fig:Snubber}
\end{figure}

\begin{enumerate}[(a)] % (a), (b), (c), ...
\item Ensures the voltage at the gate is set to 0V when the circuit is closed
\item Provide an adjustable, regulated voltage output  
\item Look nice in your firing circuit
\item Allow a path for the inductive motor current to flow when the circuit is opened
\end{enumerate}

\vfill

\end{homeworkProblem}
%%%%%%%%%%%%%%%%%%%%%%%%%%%%%%%%%%%%%%%%%%%%%%%%%%%%%%%%%%%%%%%%%%%%%%%%%%%%%%%%%%

%%%%%%%%%%%%%%%%%%%%%%%%%%%%%%%%%%%%%%%%%%%%%%%%%%%%%%%%%%%%%%%%%%%%%%%%%%%%%%%%%%
%\begin{homeworkProblem}[Question 6] % %\#\arabic{homeworkProblemCounter}]
%What is the acceptable output voltage range for the component shown in Fig.~\ref{fig:TD340} (circle one)?

%\begin{enumerate}[(a)] % (a), (b), (c), ...
%\item 5.0Vdc to 24.0Vdc
%\item 8.0Vdc to 16.0Vdc
%\item 6.5Vdc to 18.5Vdc
%\item 0.0Vdc to 5.0Vdc
%\end{enumerate}

%\vfill

%\end{homeworkProblem}
%%%%%%%%%%%%%%%%%%%%%%%%%%%%%%%%%%%%%%%%%%%%%%%%%%%%%%%%%%%%%%%%%%%%%%%%%%%%%%%%%%

%%%%%%%%%%%%%%%%%%%%%%%%%%%%%%%%%%%%%%%%%%%%%%%%%%%%%%%%%%%%%%%%%%%%%%%%%%%%%%%%%%
\begin{homeworkProblem}[Question 6] % \#\arabic{homeworkProblemCounter}]
Why did we implement switches into the firing circuit prior to implementing the mbed (circle {\bf all} that apply)?

\begin{enumerate}[(a)] % (a), (b), (c), ...
\item To test a simplified version of the circuit prior to implementing a microcontroller
\item To inject a gratuitous discussion of switches into the course
\item To confirm proper circuit wiring prior to implementing a microcontroller
\item To avoid burning out mbed I/O pins with improper circuit implementations
\end{enumerate}

\vfill

\end{homeworkProblem}
%%%%%%%%%%%%%%%%%%%%%%%%%%%%%%%%%%%%%%%%%%%%%%%%%%%%%%%%%%%%%%%%%%%%%%%%%%%%%%%%%%


%%%%%%%%%%%%%%%%%%%%%%%%%%%%%%%%%%%%%%%%%%%%%%%%%%%%%%%%%%%%%%%%%%%%%%%%%%%%%%%%%%
\begin{homeworkProblem}[Question 7] % \#\arabic{homeworkProblemCounter}]
(Short Answer) What is the purpose of the circled component in Fig.~\ref{fig:PullDownResistor}?
\begin{figure}[H]
\centering
	\includegraphics[width=2.5in]{PullDownResistor.png}
	\caption{Single MOSFET-based firing circuit with highlighted component.}
	\label{fig:PullDownResistor}
\end{figure}

\vfill
\newpage
\end{homeworkProblem}
%%%%%%%%%%%%%%%%%%%%%%%%%%%%%%%%%%%%%%%%%%%%%%%%%%%%%%%%%%%%%%%%%%%%%%%%%%%%%%%%%%

%%%%%%%%%%%%%%%%%%%%%%%%%%%%%%%%%%%%%%%%%%%%%%%%%%%%%%%%%%%%%%%%%%%%%%%%%%%%%%%%%%
\begin{homeworkProblem}[Question 8] % \#\arabic{homeworkProblemCounter}]
What type of camera are we using for the computer vision subsystem of our project?
\begin{enumerate}[(a)] % (a), (b), (c), ...
\item HD 720P 1.0MP Wide Angle Mini USB CCTV Camera
\item iCubie USB wecam
\item OV2710 high resolution micro USB camera
\item There is a camera in this project?
\end{enumerate}

\end{homeworkProblem}
%%%%%%%%%%%%%%%%%%%%%%%%%%%%%%%%%%%%%%%%%%%%%%%%%%%%%%%%%%%%%%%%%%%%%%%%%%%%%%%%%%

%%%%%%%%%%%%%%%%%%%%%%%%%%%%%%%%%%%%%%%%%%%%%%%%%%%%%%%%%%%%%%%%%%%%%%%%%%%%%%%%%%
\begin{homeworkProblem}[Question 9] % %\#\arabic{homeworkProblemCounter}]
What will happen when you close the switch shown in %Fig.~\ref{fig:ShortedFiringCircuit} (circle all that apply)?
\begin{figure}[H]
\centering
	\includegraphics[width=4.0in]{ShortedFiringCircuit.png}
	\caption{Single MOSFET-based firing circuit with a shorted-pulldown.}
	\label{fig:ShortedFiringCircuit}
\end{figure}
%
\begin{enumerate}[(a)] % (a), (b), (c), ...
\item The motor will spin faster than normal
\item The motor will not spin at all
\item The switch and connected wires may get warm or even hot
\item The motor will spin slower than normal 
\end{enumerate}

\vfill

\end{homeworkProblem}
%%%%%%%%%%%%%%%%%%%%%%%%%%%%%%%%%%%%%%%%%%%%%%%%%%%%%%%%%%%%%%%%%%%%%%%%%%%%%%%%%%

%%%%%%%%%%%%%%%%%%%%%%%%%%%%%%%%%%%%%%%%%%%%%%%%%%%%%%%%%%%%%%%%%%%%%%%%%%%%%%%%%%
\begin{homeworkProblem}[Question 10] % \#\arabic{homeworkProblemCounter}]
Explain BRIEFLY (2-3 sentences) how you processed your images in MATALB


\vspace*{\fill}
\vspace*{\fill}
\vspace*{\fill}

\end{homeworkProblem}
%%%%%%%%%%%%%%%%%%%%%%%%%%%%%%%%%%%%%%%%%%%%%%%%%%%%%%%%%%%%%%%%%%%%%%%%%%%%%%%%%%

\end{document}
