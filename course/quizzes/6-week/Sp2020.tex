\documentclass[addpoints,answers]{exam}

\newcommand{\myroot}{../../..}
\usepackage[quiz]{\myroot/course}

\title{6-week Quiz}
\author{\usnaInstructorShort}
\date{\printdate{2/12/2020}}

\begin{document}
\maketitle


\begin{questions}
\question[1]
A good problem statement for your automated turret system would accomplish the following (circle {\bf all} that apply):
\begin{choices}
\CorrectChoice Allow for design trade space and open-ended solutions
\choice Ensure the design looks exactly like the ones from the previous year  
\CorrectChoice Include any constraints or clarifying data that scope the problem
\choice Dictate the type of control algorithm required at the start of the project
\CorrectChoice Clarify both what the design will and won't do 
\CorrectChoice Consider any external interactions and influences that could affect the design
\choice Be very generic such that the objectives and design goals could be anything!
\end{choices}







\question[1]
The purpose of presenting thorough background research in an engineering report or publication is to (circle {\bf all} that apply):
\begin{choices}
\CorrectChoice Justify the contribution by comparing it to existing projects
\choice Increase the length of the document 
\CorrectChoice Convince the reader that you are not ``reinventing the wheel'' (i.e. repeating an existing and fully documented project)
\CorrectChoice Demonstrate a understanding of the field
\choice Background research serves no real purpose
\end{choices}










\question[1]\label{q:3}
What is the purpose of the component shown in \fref{fig:REG3ASW} (circle {\bf one})?
\begin{figure}[h]
\centering
	\includegraphics[width=1.5in]{MOSFET.jpg}
	\caption{50N06 MOSFET for Question~\ref{q:3}}
	\label{fig:REG3ASW}
\end{figure}
\begin{choices}
\choice Allow bidirectional control of a motor using logic level PWM and a direction pin
\CorrectChoice Acts as a switch and provides the required current ($\sim\si{\ampere}$) to power to our motors without over-drawing control inputs ($\sim\si{\milli\ampere}$)
\choice Look nice in your firing circuit
\choice Prevent current from flowing back into an mbed
\end{choices}





\clearpage
\question[1]\label{q:4}
What is the purpose of the components marked in \fref{fig:Snubber} (circle {\bf one})?
\begin{figure}[h]
\centering
	\includegraphics[width=2.3in]{Snubber_Circuit.jpg}
	\caption{Firing circuit (with button for manual testing) with important components highlighted and circled in red for Question~\ref{q:4}.}
	\label{fig:Snubber}
\end{figure}
\begin{choices}
\choice Ensures the voltage at the gate is set to \SI{0}{\volt} when the circuit is closed
\choice Provide an adjustable, regulated voltage output  
\choice Look nice in your firing circuit
\CorrectChoice Allow a path for the inductive motor current to flow when the circuit is opened
\end{choices}







%\question[1]
%What is the acceptable output voltage range for the component shown in\fref{fig:TD340} (circle one)?
%\begin{choices}
%\choice \SIrange{5.0}{24.0}{\volt}
%\choice \SIrange{8.0}{16.0}{\volt}
%\CorrectChoice \SIrange{6.5}{18.5}{\volt}
%\choice \SIrange{0.0}{5.0}{\volt}
%\end{choices}







\question[1]
Why did we implement switches (or a test wire) into the firing circuit prior to implementing the mbed (circle {\bf all} that apply)?
\begin{choices}
\CorrectChoice To test a simplified version of the circuit prior to implementing a microcontroller
\choice To inject a gratuitous discussion of switches into the course
\CorrectChoice To confirm proper circuit wiring prior to implementing a microcontroller
\CorrectChoice To avoid burning out mbed I/O pins with improper circuit implementations
\end{choices}






\question[1]\label{q:6}
Circle the \textbf{pulldown resistor} in \fref{fig:FiringCircuitPulldown}.
\begin{figure}[h]
\centering
	\includegraphics[width=4.0in]{FiringCircuit.png}
	\caption{Single MOSFET-based firing circuit for Question~\ref{q:6}.}
	\label{fig:FiringCircuitPulldown}
\end{figure}





\clearpage
\question[1]\label{q:7}
Circle the \textbf{snubber circuit} \fref{fig:FiringCircuitSnubber}
\begin{figure}[h]
\centering
	\includegraphics[width=4.0in]{FiringCircuit.png}
	\caption{Single MOSFET-based firing circuit for Question~\ref{q:7}.}
	\label{fig:FiringCircuitSnubber}
\end{figure}




\question[1]
What type of camera are we using for the computer vision subsystem of our project?
\begin{choices}
\choice HD 720P 1.0MP Wide Angle Mini USB CCTV Camera
\CorrectChoice iCubie USB wecam
\choice OV2710 high resolution micro USB camera
\choice There is a camera in this project?
\end{choices}







\question[1]\label{q:9}
What will happen when you set the mbed \lstinline{DigitalOut} pin to logic high in \fref{fig:ShortedFiringCircuit} (circle all that apply)?
\begin{figure}[h]
\centering
	\includegraphics[width=4.0in]{ShortedFiringCircuit.png}
	\caption{Single MOSFET-based firing circuit with a shorted-pulldown for Question~\ref{q:9}.}
	\label{fig:ShortedFiringCircuit}
\end{figure}
\begin{choices}
\choice The motor will spin faster than normal
\CorrectChoice The motor will not spin at all
\choice The switch and connected wires may get warm or even hot
\choice The motor will spin slower than normal 
\end{choices}






\question[1]
What is the purpose of computer vision in our \usnaCourseNumber\ project (circle all that apply)?
\begin{choices}
\choice To process images using color thresholding
\CorrectChoice To find the centroid (center) of our desired target to eventually aim the turret
\choice To process images using object properties
\choice To estimate the conversion from linear units to pixels and vice versa
\end{choices}

\end{questions}
\end{document}
