\documentclass[addpoints,answers]{exam}

\newcommand{\myroot}{../../..}
\usepackage[quiz]{\myroot/course}

\title{6-week Quiz}
\author{\usnaInstructorShort}
\date{\today}

\begin{document}
\maketitle

\begin{questions}
\question
A good problem statement for your automated turret system would accomplish all the following {\bf except} (circle all that apply):
\begin{choices}
\CorrectChoice Allow for design trade space and open-ended solutions
\choice Ensure the design looks exactly like the ones from the previous year  
\CorrectChoice Include any constraints or clarifying data that scope the problem
\choice Dictate the type of control algorithm required at the start of the project
\CorrectChoice Clarify both what the design will and won't do 
\CorrectChoice Consider any external interactions and influences that could affect the design
\choice Be very generic such that the objectives and design goals could be anything!
\end{choices}





\question
The purpose of presenting thorough background research in an engineering report or publication is to (circle {\bf all} that apply):
\begin{choices}
\CorrectChoice Demonstrate an understanding of the field
\choice Increase the length of the document 
\CorrectChoice Convince the reader that you are not repeating an existing and fully documented project
\CorrectChoice Justify the contribution by comparing it to existing projects
\choice Background research serves no real purpose
\end{choices}




\question
What is the purpose of the component shown in \fref{fig:MOSFET} (circle {\bf one})?
\begin{figure}[h]
\centering
	\includegraphics[width=2.0in]{MOSFET.jpg}
	\caption{50N06 MOSFET}
	\label{fig:MOSFET}
\end{figure}

\begin{choices}
\choice Prevent current from flowing back into an mBed
\choice Look nice in your firing circuit
\CorrectChoice Supply current on the order of \si{\ampere} to power a motor, while only drawing \si{\milli\ampere} from a control input
\choice Allow bidirectional control of a motor using logic level PWM and a direction pin
\end{choices}





\clearpage
\question
What is the absolute maximum output current rating for the component shown in \fref{fig:MOSFET} (circle {\bf one})?
\begin{choices}
\choice \SI{50}{\milli\ampere}
\choice \SI{10.0}{\ampere}
\CorrectChoice \SI{50.0}{\ampere}
\choice I have no clue how to read a datasheet
\end{choices}






\question
What is the purpose of the components shown in \fref{fig:pulldown} (circle {\bf one})?
\begin{figure}[h]
\centering
	\includegraphics[width=2.3in]{Picture4.jpg}
	\caption{Manual Firing Circuit with important components highlighted and circled in red.}
	\label{fig:pulldown}
\end{figure}
\begin{choices}
\choice Ensures the voltage at the gate is set to \SI{0}{\volt} when the circuit is closed
\choice Provide an adjustable, regulated voltage output  
\choice Look nice in your firing circuit
\CorrectChoice Allow a forward path for the inductive motor current to flow when the circuit is opened
\end{choices}






%\question
%What is the acceptable output voltage range for the component shown in Fig.~\ref{fig:TD340} (circle one)?
%\begin{choices}
%\choice \SIrange{5.0}{24.0}{\volt}
%\choice \SIrange{8.0}{16.0}{\volt}
%\choice \SIrange{6.5}{18.5}{\volt}
%\choice \SIrange{0.0}{5.0}{\volt}
%\end{choices}






\question
Why did we implement switches (or a test wire) into the firing circuit prior to implementing the mbed (circle {\bf all} that apply)?
\begin{choices}
\CorrectChoice To test a simplified version of the circuit prior to implementing a microcontroller
\choice To inject a gratuitous discussion of switches into the course
\CorrectChoice To confirm proper circuit wiring prior to implementing a microcontroller
\CorrectChoice To avoid burning out mbed I/O pins with improper circuit implementations
\end{choices}






%\question
%For a given range, if you calculate a CEP associated with a 75\% probability and take 100 shots, approximately how many shots will hit within the CEP?
%\begin{solution}[3in]
%The expectation $E(x) = 0.75\times 100 = 75$ shots. 
%\end{solution}




\clearpage
\question
What is the purpose of the circled component in \fref{fig:PullDownResistor}?
\begin{figure}[h]
\centering
	\includegraphics[width=2.5in]{PullDownResistor.png}
	\caption{Single MOSFET-based firing circuit with highlighted component.}
	\label{fig:PullDownResistor}
\end{figure}
\begin{solution}[3in]
The circled component is a pulldown resistor, it ensures the MOSFET turns off when nothing else is connected to the gate. 
\end{solution}
% Sp2020: Mike is asking to make this multiple choice




\question
What type of camera are we using for the computer vision subsystem of our project?
\begin{choices}
\choice HD 720P 1.0MP Wide Angle Mini USB CCTV Camera
\CorrectChoice iCubie USB wecam
\choice OV2710 high resolution micro USB camera
\choice There is a camera in this project?
\end{choices}



\clearpage
\question
What will happen when you close the switch shown in \fref{fig:ShortedFiringCircuit} (circle all that apply)?
\begin{figure}[h]
\centering
	\includegraphics[width=4.0in]{ShortedFiringCircuit.png}
	\caption{Single MOSFET-based firing circuit with a shorted pulldown.}
	\label{fig:ShortedFiringCircuit}
\end{figure}
\begin{choices}
\choice The motor will spin faster than normal
\CorrectChoice The motor will not spin at all
\choice The switch and connected wires may get warm or even hot
\choice The motor will spin slower than normal 
\end{choices}




\question
Explain briefly (2-3 sentences) how you processed your images in \Matlab. 
\begin{solution}[3in]
\end{solution}

\end{questions}
\end{document}