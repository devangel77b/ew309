\documentclass[aspectratio=169]{beamer}
\usepackage[utf8]{inputenc}
\usepackage[scaled]{helvet}
\renewcommand\familydefault{\sfdefault}
\usepackage[T1]{fontenc}

\newcommand{\myroot}{../..}
\usepackage[slides]{\myroot/course}
\title{Deriving a transfer function from a step response}
\subtitle{\usnaCourseNumber\ Guided Design Experience, \usnaCourseTerm}
\author{\usnaInstructorShort}
\date{\today}
	
\usetheme{Hopper}

\begin{document}
\settitlebg
\begin{frame}
\titlepage
\end{frame}

\setslidebg
\begin{frame}
\frametitle{Step response review}
\end{frame}

\begin{frame}
\frametitle{Modeling the truret}
\end{frame}
% with sequential build

\begin{frame}
\frametitle{Findings}
\end{frame}

\begin{frame}
\frametitle{Findings}
\end{frame}

\begin{frame}
\frametitle{Estimating $a$ and $b$}
\end{frame}

\begin{frame}
\frametitle{Using \Matlab to find coefficients for a known function}
\begin{columns}
\begin{column}{0.5\textwidth}
Graphic here
\end{column}
\begin{column}{0.5\textwidth}
\begin{enumerate}
\item Collect data
%\begin{lstlisting}[style=usnaMatlab]
%t = ; % Data collected from mbed (1xN array)
%theta_dot = % Data collected from mbed (1xN array)
%\end{lstlisting}
\item Define the function
%\begin{lstlisting}[style=usnaMatlab]
%fittype_measured = fittype(...
%	'( (0.5*b)/a )*(1 - exp(-a*x))',... 	'dependent',{'y'},... 
%	'independent',{'x'},... 	'coefficients',{'a','b'});
%\end{lstlisting}
\item Fit the function
%\begin{lstlisting}[style=usnaMatlab]
%fit_measured = fit(...
%	t',theta_dot',...
%	fittype_measured,...
%	'StartPoint',[1,1],...
%	'Lower', [0, 0]);
%\end{lstlisting}
\end{enumerate}
\end{column}
\end{columns}
\end{frame}

\begin{frame}
\frametitle{Accounting for friction}
\end{frame}
% and sequential build

\begin{frame}
\frametitle{Using \Matlab\ to find poles and zeros for a transfer function directly}
\end{frame}

\begin{frame}[fragile]
\frametitle{Format the data}
The \lstinline{iddata} function is required when using \Matlab's transfer function estimation tools. The following syntax should work for your application:
\begin{lstlisting}[style=usnaMatlab]
data_pos = iddata(...
	uniform_theta,...	% Measured output  
	uniform_PWM,...	% Applied input
	[],...	% Sampling frequency (empty if you supply sample times)
	'SamplingInstants',uniform_t,...	% Sample times
	'InputName',{'Signed PWM'},...	% Specify input name(s)       (optional)
	'InputUnit',{'unitless'},...	% Specify units of input(s)   (optional)
	'OutputName',{'Turret Angle'},...	% Specify output name(s)      (optional)
	'OutputUnit',{'deg'});	% Specify units for output(s) (optional)
\end{lstlisting}
\end{frame}

\begin{frame}[fragile]
\frametitle{Estimate the transfer function}
The \lstinline{tfest} function can estimate a specified number of poles and zeros given fit conditions specified by \lstinline{tfestOptions}
\begin{lstlisting}[style=usnaMatlab]
% Define number of poles
np = 2;
% Define number of zeros
nz = 0;
% Define transfer function fit options
opt = tfestOptions('InitMethod','all','InitialCondition','zero');
% Calculate plant transfer function
H = tfest(data_pos,np,nz,opt);
\end{lstlisting}
\end{frame}

\begin{frame}[fragile]
\frametitle{Plot the transfer function}
As with all of your work, visualizing your results is important.
Along with the plotted data collected from the mbed, you can plot the fitted transfer function using the following
\begin{lstlisting}[style=usnaMatlab]
% Find where step is initiated
idx = find(uniform_PWM == 0, 1, 'Last');
idx_Start = idx+1;  % Index value where step input is applied
% Define fixed PWM value of step
PWM_val = mean(uniform_PWM(idx_Start:end));
 
% Position response
[theta_estimate,t_out] = step(PWM_val*H,...
    uniform_t(idx_Start:end) - uniform_t(idx_Start));
% Plot response
plot(t_out + uniform_t(idx_Start),theta_estimate,'g','LineWidth',2);
\end{lstlisting}
\end{frame}

\begin{frame}[fragile]
\frametitle{Get the coefficients of the transfer function}
The information including the poles and zeros of the transfer function are stored in an ``identified transfer function'' object in \Matlab. You can parse the numerator and denominator coefficients of the transfer function as follows:
\begin{lstlisting}[style=usnaMatlab]
% Parse information from the identified transfer function object
pvec = getpvec(H,'free');
b = pvec(1);
c(1) = pvec(2);
c(2) = pvec(3);
\end{lstlisting}
\end{frame}

\begin{frame}[fragile]
\frametitle{Derive the poles of the transfer function}
Define the numerator and poles of the transfer function:
\begin{lstlisting}[style=usnaMatlab]
% Calculate the poles using algebraic manipulation
syms x
fcn = x^2 + c(1)*x + c(2);
factored_fcn = factor(fcn, x, 'FactorMode', 'real');
for i = 1:numel(factored_fcn)
    coeffs_fcn = coeffs(factored_fcn(i));
    p(i) = coeffs_fcn(1);
end
% Label new coefficients
delta = p(1);
a = p(2);
 
% Displace transfer function
fprintf('Estimated Transfer function:\n');
fprintf('\t H(s) = %f / ( (s + %f)*(s + %f) ) \n',b,delta,a);
\end{lstlisting}
\end{frame}

\begin{frame}
\frametitle{Implementation notes}
If you choose to fit the transfer function to the position data (which is recommended), be sure to include ``equilibrium values'' in your data
This means you want to start collecting data before you start moving your motor!
\end{frame}

\begin{frame}[fragile]
\frametitle{mbed data collection pseudocode}
\begin{lstlisting}[style=mbedC]
float t = 0.00;
float dt = 0.02;
float PWM = 0.00;
PWMout = PWM;
for( t = 0.00; t <= 2.0; t = t + dt ){
	turretAngle = get turret angle;
	printf("%.3f,%.3f,%.3f\n",t,PWM,turretAngle);
}

float t_now = t;
PWM = 0.50;
PWMout = PWM;
for( t = t_now; t <= 10.0; t = t + dt ){
	turretAngle = get turret angle;
	printf("%.3f,%.3f,%.3f\n",t,PWM,turretAngle);
}
\end{lstlisting}
\end{frame}

\end{document}
