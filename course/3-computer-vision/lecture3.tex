\documentclass[aspectratio=169]{beamer}
\usepackage[utf8]{inputenc}
\usepackage[scaled]{helvet}
\renewcommand\familydefault{\sfdefault}
\usepackage[T1]{fontenc}

\newcommand{\myroot}{../..}
\usepackage[slides]{\myroot/course}
\title{Vision based targeting}
\subtitle{\usnaCourseNumber\ Guided Design Experience, \usnaCourseTerm}
\author{\usnaInstructorShort}
\date{\today}

\RequirePackage{tikz}
\usetikzlibrary{shapes,arrows}
\tikzset{
	arrow/.style={thick,->,>=stealth},
	noarrow/.style={thick},
	block/.style = {draw,rectangle, minimum height=3em, minimum width=6em},
	sum/.style = {draw, circle, node distance=2cm},
	input/.style = {coordinate},
	output/.style = {coordinate},
	pinstyle/.style = {pin edge={to-, thin, black}},
	}
	
\usetheme{Hopper}

\begin{document}
\settitlebg
\frame{\titlepage}

\setslidebg
\begin{frame}
\frametitle{Motivation}
\framesubtitle{Find the yellow, red, green, orange, or white buoy, line, or ball}
\includegraphics[height=2cm]{\myroot/figures/l3-f2a.png}
\includegraphics[height=2cm]{\myroot/figures/l3-f2b.png}
\includegraphics[height=2cm]{\myroot/figures/l3-f2c.png}
\includegraphics[height=2cm]{\myroot/figures/l3-f2d.png}

Problem: The real world...

\includegraphics[height=2cm]{\myroot/figures/l3-f2e.png}
\includegraphics[height=2cm]{\myroot/figures/l3-f2f.png}
\includegraphics[height=2cm]{\myroot/figures/l3-f2g.png}
\includegraphics[height=2cm]{\myroot/figures/l3-f2h.png}
\end{frame}

\begin{frame}[label=bigpicture]
\frametitle{The big picture}
\begin{center}
\begin{tikzpicture}[auto, >=latex']
\node[block] (input) {target angle};
\node[sum, right of=input] (sum) {};
\node[block, right of=sum, node distance=2cm,align=center] (controller) {controller\\$KG_c(s)$};
\node[block, right of=controller, node distance=3cm,align=center] (plant) {plant\\$G_{sys}(s)$};
\node[output, right of=plant, node distance=2cm] (output) {};

\draw [arrow] (input) -- node [pos=0.95, above] {$+$} (sum); 
\draw[arrow] (sum) -- (controller);
\draw[arrow] (controller) -- node [pos=0.5, name=x] {} (plant);
\draw[noarrow] (plant) -- (output); 

\node[block, below of=x, node distance=1.5cm,align=center](sensor) {position\\sensor};
\draw[arrow] (output) |- (sensor);
\draw[arrow] (sensor) -| node [pos=0.95, left] {$-$} (sum);

\node [above of=controller, node distance=1.5cm] (controllerpic) {\includegraphics[width=2cm]{\myroot/figures/l3-f3controller.png}};
\node [above of=plant, node distance=1.8cm] (plantpic) {\includegraphics[width=2cm]{\myroot/figures/l3-f3plant.png}};
\node [below of=sensor, node distance=1.25cm] (sensorpic) {\includegraphics[width=2cm]{\myroot/figures/l3-f3sensor.png}};
\node [below of=input, node distance=2.5cm] (targetpica) {\includegraphics[width=1cm]{\myroot/figures/l3-f3targeta.png}};
\node [below of=input, node distance=1.3cm] (targetpicb) {\includegraphics[width=2cm]{\myroot/figures/l3-f3targetb.png}};

\end{tikzpicture}
\end{center}
\end{frame}


\frame{\frametitle{The big picture goal}
\begin{enumerate}
\item Use our camera to capture images (Part 1 - Acquiring images)
\item Process those images to filter by (Part 2 - Image procesing)
\begin{itemize}
\item color
\item shape
\item size
\end{itemize}

Once we have a final filtered image, hopefully with a single target, 
\item Calculate the centroid of the target in SI units (Part 3 - Estimating scale)
\item Test our procedure before implementing on real weapon (Part 4 - Testing and validation)
\end{enumerate}
}

\frame{\frametitle{What you see...}
\begin{center}
\includegraphics[height=7cm]{\myroot/figures/l3-f5.png}
\end{center}
}

\frame{\frametitle{What your computer sees!}
\begin{center}
\includegraphics[height=7cm]{\myroot/figures/l3-f6.png}
\end{center}
}

\frame{\frametitle{Grayscale images are stored as arrays (matrices)}
\begin{columns}
\begin{column}{0.5\textwidth}
\begin{center}
\includegraphics[width=\columnwidth]{\myroot/figures/l3-f6.png}
\end{center}
\end{column}
\begin{column}{0.5\textwidth}
\begin{itemize}
\item 640 columns by 480 rows, or about 1/3 mega pixels
\item unsigned 8-bit integer (\lstinline{uint8}), 0-255
\item In \Matlab, if the image is stored as a variable \lstinline{im}, individual pixel intensity/brightness can be referenced using \lstinline{im(8,6) = 34}. 
\end{itemize}
\end{column}
\end{columns}
}

\frame{\frametitle{Color images}
\begin{columns}
\begin{column}{0.4\textwidth}
\begin{center}
\includegraphics[width=\columnwidth]{\myroot/figures/l3-f8rgb.png}
\end{center}
\end{column}
\begin{column}{0.1\textwidth}
\begin{center}
\includegraphics[width=\columnwidth]{\myroot/figures/l3-f8r.png}\\
\vspace{0.4cm}
\includegraphics[width=\columnwidth]{\myroot/figures/l3-f8g.png}\\
\vspace{0.4cm}
\includegraphics[width=\columnwidth]{\myroot/figures/l3-f8b.png}

\end{center}
\end{column}

\begin{column}{0.5\textwidth}
Even though the block looks ``red'' the block is really a combination of RGB values.
\begin{itemize}
\item Color images are stored as a 3-D matrix, e.g. shape \num{640x480x3}.
\item Third channel is called color channel or bit plane (e.g. red, green, or blue channel)
\item Each color channel can be 0-255, white is (255,255,255), black is (0,0,0).
\item The red block is about (200,75,20), so ``red'' is not necessarily all R. 
\end{itemize}
\end{column}
\end{columns}
}

\frame{\frametitle{How to accomplish our big picture goal with the USB camera}
\begin{center}
\includegraphics[width=\columnwidth]{\myroot/figures/l3-f9.png}
%
%\includegraphics[width=3cm]{\myroot/figures/l3-f9b.png}
\end{center}
}

\frame{\frametitle{What can go wrong?}
}

\frame{\frametitle{There is only so far you can get with color alone}
\begin{center}
\includegraphics[height=7cm]{\myroot/figures/l3-f11.png}
\end{center}
}

\frame{\frametitle{Choose your region properties wisely}
\begin{center}
\includegraphics[width=0.5\columnwidth]{\myroot/figures/l3-f12.png}
\end{center}
Region properties divide a binary image into contiguous regions. Compute numerical properties such as
\begin{itemize}
\item area (in pixels)
\item centroid (in pixels)
\item eccentricity (ratio of the distance between the foci and the major axis length)
\end{itemize}
\hfill\includegraphics[width=2cm]{\myroot/figures/l3-f12b.png}
}

\frame{\frametitle{Contrast makes life eas(ier)}
\begin{columns}
\begin{column}{0.5\textwidth}
\begin{center}
Easy\\
\includegraphics[width=0.5\columnwidth]{\myroot/figures/l3-f13a.png}
\end{center}
Controlled environment. Uniform color object. Background very different from object. Background is homogeneous and uncluttered. 
\end{column}
\begin{column}{0.5\textwidth}
\begin{center}
Hard\\
\includegraphics[width=0.5\columnwidth]{\myroot/figures/l3-f13b.png}
\includegraphics[width=0.33\columnwidth]{\myroot/figures/l3-f13c.png}
\includegraphics[width=0.33\columnwidth]{\myroot/figures/l3-f13d.png}
\end{center}
\end{column}
\end{columns}
}

\againframe{bigpicture}

\frame{\frametitle{Slow update rate}
\begin{itemize}
\item Acquiring \SI{30}{frame\per\second} (\SI{30}{\hertz}) is considered full video
\item Some can go as fast as \SI{60}{\hertz}. Special purpose cameras might go as fast as \SI{1000}{\hertz}
\item Processing takes time; if you get \SI{10}{\hertz} in \usnaCourseNumber\, that is good.
\item For comparison:
\begin{itemize}
\item Industrial robot controller: \SI{500}{\hertz}
\item Inertial measurement for guidance \SI{2000}{\hertz}
\end{itemize}
\end{itemize}
}

\end{document}
