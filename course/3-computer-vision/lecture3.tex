\documentclass[aspectratio=169]{beamer}

\newcommand{\myroot}{../..}
\usepackage[slides]{\myroot/course}
\title{Vision based targeting}
\subtitle{\usnaCourseNumber\ Guided Design Experience, \usnaCourseTerm}
\author{\usnaInstructorShort}
\date{\today}

\usetheme{Hopper}

\begin{document}
\settitlebg
\frame{\titlepage}

\setslidebg
\begin{frame}
\frametitle{Motivation}
\framesubtitle{Find the yellow, red, green, orange, or white buoy, line, or ball}
Problem: The real world...
\end{frame}

\frame{\frametitle{The big picture}
}

\frame{\frametitle{The big picture goal}
\begin{enumerate}
\item Use our camera to capture images (Part 1 - Acquiring images)
\item Process those images to filter by (Part 2 - Image procesing)
\begin{itemize}
\item color
\item shape
\item size
\end{itemize}

Once we have a final filtered image, hopefully with a single target, 
\item Calculate the centroid of the target in SI units (Part 3 - Estimating scale)
\item Test our proceudre before implementing on real weapon (Part 4 - Testing and validation)
\end{enumerate}
}

\frame{\frametitle{What you see...}
}

\frame{\frametitle{What your computer sees!}
}

\frame{\frametitle{Grayscale images are stored as arrays (matrices)}
\begin{columns}
\begin{column}{0.5\textwidth}
\end{column}
\begin{column}{0.5\textwidth}
\begin{itemize}
\item 640 columns by 480 rows, or about 1/3 mega pixels
\item unsigned 8-bit integer (\lstinline{uint8}), 0-255
\item In \Matlab, if the image is stored as a variable \lstinline{im}, individual pixel intensity/brightness can be referenced using \lstinline{im(8,6) = 34}. 
\end{itemize}
\end{column}
\end{columns}
}

\frame{\frametitle{Color images}
\begin{columns}
\begin{column}{0.5\textwidth}
\end{column}
\begin{column}{0.5\textwidth}
Even though the block looks ``red'' the block is really a combination of RGB values.
\begin{itemize}
\item Color images are stored as a 3-D matrix, e.g. shape \num{640x480x3}.
\item Third channel is called color channel or bit plane (e.g. red, green, or blue channel)
\item Each color channel can be 0-255, white is (255,255,255), black is (0,0,0).
\item The red block is about (200,75,20), so ``red'' is not necessarily all R. 
\end{itemize}
\end{column}
\end{columns}
}

\frame{\frametitle{How to accomplish our big picture goal with the USB camera}
}

\frame{\frametitle{What can go wrong?}
}

\frame{\frametitle{There is only so far you can get with color alone}
}

\frame{\frametitle{Choose your region properties wisely}
Region properties divide a binary image into contiguous regions. Compute numerical properties such as
\begin{itemize}
\item area (in pixels)
\item centroid (in pixels)
\item eccentricity (ratio of the distance between the foci and the major axis length)
\end{itemize}
}

\frame{\frametitle{Contrast makes life eas(ier)}
\begin{columns}
\begin{column}{0.5\textwidth}
Easy: controlled enviornment. Uniform color object. Background very different from object. Background is homogeneous and uncluttered. 
\end{column}
\begin{column}{0.5\textwidth}
Hard. 
\end{column}
\end{columns}
}

\frame{\frametitle{The big picture}
}

\frame{\frametitle{Slow update rate}
\begin{itemize}
\item Acquiring \SI{30}{frame\per\second} (\SI{30}{\hertz}) is considered full video
\item Some can go as fast as \SI{60}{\hertz}. Special purpose cameras might go as fast as \SI{1000}{\hertz}
\item Processing takes time; if you get \SI{10}{\hertz} in \usnaCourseNumber\, that is good.
\item For comparison:
\begin{itemize}
\item Industrial robot controller: \SI{500}{\hertz}
\item Inertial measurement for guidance \SI{2000}{\hertz}
\end{itemize}
\end{itemize}
}

\end{document}
