\documentclass[aspectratio=169]{beamer}
\usepackage[utf8]{inputenc}
\usepackage[scaled]{helvet}
\renewcommand\familydefault{\sfdefault}
\usepackage[T1]{fontenc}

\newcommand{\myroot}{../../..}
\usepackage[slides]{\myroot/course}
\title{Sizing and selecting a motor}
\subtitle{\usnaCourseNumber\ Guided Design Experience, \usnaCourseTerm}
\author{\usnaInstructorShort}
\date{\today}
	
\usetheme{Hopper}

%\usepackage{siunitx}
%\DeclareSIUnit{\inch}{in}

\begin{document}
\settitlebg
\begin{frame}
\titlepage
\end{frame}

\setslidebg
\begin{frame}
\frametitle{Functional requirements}
Brainstorm a list of things the turret motor must be able to do. 
Be as specific as possible.
Circle things that need additional computation.
\end{frame}

\begin{frame}
\frametitle{Functional requirements: speed}
Draw a simple speed profile (velocity versus time) for a move of \ang{30} in \SI{2}{\second}. What is the maximum acceleration for your profile? Why is this important? 
\end{frame}

\begin{frame}
\frametitle{Functional requirements: torque}
List all items that contribute to the motor's load. What equations are relevant? 
\end{frame}

\begin{frame}
\frametitle{Functional requirements: torque}
Using the spreadsheet provided on the Google Drive, estimate the torque required to move your turret. 
Use careful measurements (to nearest \si{\gram}, \SI{0.125}{\inch}, etc)
Convert all units to SI (\si{\kilo\gram} and \si{\meter})
\end{frame}

\begin{frame}
\frametitle{Design margin}
What are the possible consequences if our estimates are too low?
What are possible consequences if our estimates are too high?
Torque estimate with design margin
% A designer should aim for a 50\% torque allowance for most industrial applications \cite{stiffler-1992-design}
\end{frame}
% Stiffler, Design with Microprocessors for Mechanical Engineers

\begin{frame}
\frametitle{What is the power required?}
Power required
\begin{equation}
P = T_f\omega + J\alpha\omega
\end{equation}
Motor power
\begin{equation}
P = 
\end{equation}
%As a starting point, choose a motor with double the calculated power requirements \cite{stiffler-1992-design}
\end{frame}


\begin{frame}
\frametitle{So let's pick a motor!}
What kind of motors have you learned about or can we use for our auto turret application?
\end{frame}

\begin{frame}
\frametitle{Simplified list of DC motors}
Brushed DC motor
Brushless DC motor
Stepper motor
(industrial) Servo motor
Servo
\end{frame}

\begin{frame}
\frametitle{Brushed DC motor}
Motor typically is controlled using two wires
Feedback control typically requires an encoder
Motors are typically sold with a gearbox/gearhead
Brushed DC Motor Assembly typically includes
Brushed DC Motor
Encoder
Gearbox/Gearhead
\end{frame}

\begin{frame}
\frametitle{Brushless DC motor}
Motor is controlled using multiple wire pole pairs
Feedback control typically requires an encoder
Motors are typically sold with a gearbox/gearhead
``Brushless DC Motor Assembly'' typically includes
Brushless DC Motor
Encoder
Gearbox/Gearhead
\end{frame}

\begin{frame}
\frametitle{Stepper motor}
Motor is controlled using multiple wire pole pairs
Motor is typically controlled open-loop using a ``step'' process
Motors are typically controlled using a commercial-off-the-shelf stepper motor controller rated for the specific motor.
\end{frame}

\begin{frame}
\frametitle{(industrial) Servo motor}
The ``motor'' is actually an assembly containing a DC motor, typically a gearbox, and sensor (e.g. encoder).
Servo motors often require a ``control module'' to interface with a system.
Control modules are specific to the motor assembly and typically act as a closed-loop position controller.
\end{frame}

\begin{frame}
\frametitle{Servo}
The ``motor'' is actually a closed-loop system containing a DC motor, gearbox, potentiometer, and feedback control circuit
Motors respond to a common pulse coded signal to drive to desired positions
RC servos typically have rotation constraints
\end{frame}

\begin{frame}
\frametitle{How to decide a motor for your auto turret}
Select a set of objectives (goals for your design) and then prioritize them
Some examples might be (3) cost, (2) weight, and (1) size
The bigger the number the higher the priority
Establish some conceptual metrics for each objective
For cost – you might have less than \$100 (5 pts), between \$100-\$1000 (3 pts), or over \$100 (0 pts).  For weight you might have less than \SI{400}{\gram} (5pts) , \SIrange{400}{1000}{\gram} (3pts), over \SI{1000}{\gram} (1 pt). 
For each motor alternative, use your metrics to score each objective 
Multiplying the objective weight (priority) by the metric score and then sum all the weighted objective scores
``Best'' design is that with the highest composite score
\end{frame}

\begin{frame}
\frametitle{Constraints}
You must pick design alternatives that meet your auto-turret requirements
Otherwise you would not be considering that motor
You also must pick design alternatives that are feasible
The motor that drives the Lone Star's geothermal drill and water well is super cool since it drills as deep as \SI{400}{\foot} through hard rock and volcanic formations.  But chances are it is probably overkill for your nerf turret.
\end{frame}

\begin{frame}
\frametitle{Autonomous turret system decision matrix}
\end{frame}

\begin{frame}
\frametitle{To do}
Verify that the DC motor in the turret meets our requirements 
Use the Excel spreadsheet to calculate torque requirements
Check speed, torque, and power at our operating voltage
Identify an alternative motor that would fulfill our requirements
Website search using online datasheets
Come up with a few objectives, weight (prioritize) them, and generate metrics for your objectives as a project team for your turret system
Score the two motors using the weights to compute totals. 
Be prepared to discuss your choice in your final report
Complete Moving the turret (instructions) document.
\end{frame}
\end{document}
