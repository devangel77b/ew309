\documentclass[aspectratio=169]{beamer}
\usepackage[utf8]{inputenc}
\usepackage[scaled]{helvet}
\renewcommand\familydefault{\sfdefault}
\usepackage[T1]{fontenc}

\newcommand{\myroot}{../..}
\usepackage[slides]{\myroot/course}
\title{Firing the nerf guns with mbed}
\subtitle{\usnaCourseNumber\ Guided Design Experience, \usnaCourseTerm}
\author{\usnaInstructorShort}
\date{\today}
	
\usetheme{Hopper}

\begin{document}
\settitlebg
\begin{frame}
\titlepage
\end{frame}

\setslidebg
\begin{frame}
\frametitle{General instructions (1 of 2)}
\begin{itemize}
\item If you decide to leave the mechanical switches in your firing circuit, you must disconnect them COMPLETELY before connecting the mBed
\begin{itemize}
\item This prevents potential damage to your microcontroller
\end{itemize}
\item Mount your mBed on the protoboard.  Be sure to place it such that the USB connector is accessible from the end of the protoboard (if possible)
\item Connect the ground pin of the mBed to the circuit GND on your breadboard
\begin{itemize}
\item Be aware of breaks in the protoboard – install jumpers as needed)
\end{itemize}
\end{itemize}
\end{frame}

\begin{frame}
\frametitle{General instructions (1 of 2)}
\begin{itemize}
\item Select two \lstinline{PwmOut} pins on the mBed to connect to your the gates of your 50N06 MOSFETs.
\begin{itemize}
\item The \SI{10}{\kilo\ohm} pull down resistors remain in the circuit!
\end{itemize}
\item Write a program that will:
\begin{itemize}
\item Fire a prescribed number of shots (remember sequencing of flywheel and pusher motors)
\item Associate an LED to a motor and turn on that LED when that motor is supposed to be on (there are four LEDs available).  This will be very useful for debugging
\end{itemize}
\end{itemize}
\end{frame}

\begin{frame}
Graphic here
\end{frame}

\end{document}
