\documentclass[11pt,courier]{navymemo}

%\newcommand{\myroot}{../..}
%\usepackage{\myroot/ew305}
%incompatible for some reason?

\newcommand{\usnaCourseNumber}{EW309}
\newcommand{\usnaCourseName}{Guided Design Experience}
\newcommand{\usnaInstructor}{Assistant Professor D Evangelista}
\newcommand{\usnaInstructorShort}{D Evangelista}
\newcommand{\usnaCourseSection}{0111 0311}
\newcommand{\usnaCourseTerm}{Spring 2020}
\newcommand{\Google}{Google}
\newcommand{\GoogleClassroom}{Google Classroom}

\author{\usnaInstructorShort}
\title{\usnaCourseNumber\ Course Policy}
\navysubj{Course policy for \usnaCourseNumber\ (\usnaCourseName)}
\navyfiling{1531}
%\navyserial{16-0001}
\date{\today}
%\navymarking{UNCLASSIFIED}

\usepackage{designature}
%\usepackage{siunitx}
\usepackage[colorlinks=true,urlcolor=blue]{hyperref}
%\usepackage{gitinfo2}

\usepackage{multirow}
\usepackage{xcolor,colortbl}
\usepackage{hhline}

\begin{document}
\makedateblock{}

\MEMORANDUM{}

\begin{navyletterheader}
\navyfrom{\usnaInstructor}
\navyto{\usnaCourseNumber\ Section \usnaCourseSection}
\navyskip{}%

\navysubjline{}
\navyskip{}%
\navyref{refa}{ACDEANINST 1531.58 (Administration of Academic Programs)}
\navyref{refb}{USNAINST 1531.53B (Policies Concerning Graded Academic Work)}
\navyref{refc}{USNAINST 1610.3H (Brigade Honor Program)}
\navyref{refd}{COMDTMIDNINST 1752.1E (Sexual Assault Prevention and Response Program)}
%\navyskip{}%
\navyencl{encl1}{Syllabus, \usnaCourseNumber\ (\usnaCourseName), \usnaCourseTerm}
\end{navyletterheader}

\section{} Per reference~(\ref{refa}), this memorandum sets forth my course policy for the \usnaCourseTerm\ session of \usnaCourseNumber\ (\usnaCourseName). This information supplements the basic guidance provided in references~(\ref{refa}) through (\ref{refd}) and enclosure~(\ref{encl1}).
 
\section{Course objectives} At the end of the course, a student should be able to:
\begin{itemize}
\setlength\itemsep{0em}
\item Develop and execute a formal test plan to acquire experimental data.
\item Apply appropriate analysis techniques for interpretation of measured data.
\item Perform appropriate research on components to span a design space.
\item Make informed decisions for parts selection.
\item Analyze and design a control system to meet the needs of a project.
\item Apply basic computer vision methods. % through \Matlab.
\end{itemize}
 
Additionally, the course is intended to:
\begin{itemize}
\item Foster a preliminary understanding of and appreciation for the engineering design process.
\item Provide a framework for exercise and integration of the components of the robotics and control engineering curriculum into a complete project.
\item Enhance hands-on skills.
\end{itemize}
 
\section{Textbook} There is no textbook for this course.
 
\section{Grading policy} The interim and final course grades will be based on the following approximate grade weights:
\begin{table}[h]
\begin{center}
\begin{tabular}{|l|l|}\hline
Individual assessment (individual assignments, quizzes) & 20\% \\\hline
Interim project deliverables (lab assignments done as a team) & 30\% \\ \hline
Final project demonstration & 20\% \\ \hline
Final design report & 20\% \\ \hline
Professionalism (class participation, lab/workspace neatness) & 10\% \\ \hline
\end{tabular}
\end{center}
\end{table}

\section{Assignments} Lab assignments will be in groups of two (typically) or three (rarely). Reports will follow the standard department template, to be provided. Late lab reports will face a penalty of 10\% per day unless otherwise coordinated with the instructor. All sources of information and any collaboration must be fully documented.
 
Individual assignments are to be completed without collaboration between midshipmen unless otherwise stated. All sources of information must be fully documented. Late individual assignments accrue a 10\% penalty per day unless otherwise coordinated with the instructor.
 
Putzier rule: you must get a passing grade on the final project demonstration and design report in order to pass the class.

See reference~(\ref{refc}) for more information on the standards of academic rigor that will be universally applied to work in this class.
 
\section{Classroom and laboratory hardware.}
You must abide by all posted instructions in regards to hardware used in class and in laboratories. Disobeying posted instructions or those given in laboratory assignments, tampering with hardware used by other groups and/or moving hardware assigned to specific stations is forbidden and will result in a reduction of grade for your group on the assignment. You will be assigned a lock and a locker. You may not replace the lock with your own, and you must not block the view of the contents of your locker. Every instructor has the master list of combinations and a master key.
 
\section{Classroom work and assigned reading}
I will assume that you are prepared for class. In this regard, I expect any assigned reading, as delineated in class, to be completed prior to the appropriate class session. You are responsible for all text and other handout material.
 
\section{Classroom attendance}
Class attendance is a prerequisite to success. If you anticipate an excused absence, let me know in advance. Your responsibility is to turn in any course work when due and to obtain notes and announcements from another class member for classes you have missed.
 
\section{Classroom work and decorum}
\begin{itemize}
\item Sleeping during class is unacceptable and will result in significant grade reduction, especially if chronic.
\item Treat everyone in the class with dignity and respect. Respect the viewpoints of your peers and do not disparage their efforts, but rather offer constructive criticism where appropriate and in a professional manner.
\item PLEASE KEEP YOUR WORKSPACE TIDY. Please dispose of your trash.
\item Please store your backpacks outside on hooks or against the wall. This will help open up space between desks and prevent others (including myself) from tripping over them.
\item Per reference~(\ref{refd}), an important aspect in the implementation of the Naval Academy’s mission is to develop selfless leaders who value diversity and create an ethical command climate through their example of personal integrity and moral courage. If this attribute remains a common goal for each Midshipman, faculty and staff, all members of the team will be treated with dignity and respect. Additionally, the Brigade and classroom climate will not tolerate violence against others, reducing the potential risk of sexual assault.
\end{itemize}

\section{Absent/late instructor}
In the event that the course instructor is more than ten minutes late for class, the section leader will report to the department office (Maury 306) for further instruction. If there is no packet, and no further instructions are available from the department educational technician, the remainder of the class should be used to work on the assigned project or homework assignment.  Class is not canceled.
 
\section{Formats/signatures}
Most of the work in \usnaCourseNumber\ should be electronically generated. Work should be submitted as directed. All work must be neat,  and legible. If you do not put your name on the submission, you will not receive credit for the work; just putting it in your \Google\ drive folder or \GoogleClassroom\ is not enough).
 
\section{Communication/extra instruction (EI)} Any time you are having difficulty with the course material, I encourage you to seek EI. Extra instruction for this course is best scheduled by appointment so we can work together in the lab on anything you are struggling with. My primary means of communication with you outside the classroom will be via email -- so check it frequently.

\section{\GoogleClassroom\ Folder}
I will create a \GoogleClassroom\ for this course.  Since \GoogleClassroom\ creates a separate profile for each student, for those assignments that require working together as a team, I will create shared assignments.  This should allow for collaboration when it is allowed and for individual work when it is not.   Each group will be required to submit their individual or group assignments using \GoogleClassroom\ by the due date.  One the due date has passed, you are not allowed to make any changes to your assignments until you have received a grade or the instructor has given you permission to make changes.  For those of you who haven’t used \GoogleClassroom\ before, I will spend some time in the first week going over it so you can become familiar with how it works.

\section{Other software} Use of Latex for report preparation, R for statistical analysis, Gimp and Inkscape for figure preparation, revision control via Git, and Scrum/agile project management via Trello are all encouraged. 

\clearpage
\section{Office Hours and Extra Instruction (EI)} Midshipmen are encouraged to arrange EI as needed (appointments encouraged). I am teaching EW309 (Guided Design Experience) MWF1-4; EW282C (School of Drones) R5-6; and EW282D (Biomechanics Lab) F3-4. Tuesday is typically used for research but is available by appointment. There may be a Labrador retriever guide dog puppy in training at my office; if this is an issue for you I can make alternate arrangements. 
\begin{table}[h]
\begin{center}
\begin{tabular}[h]{| c | c | c | c | c | c | c |}
\hline
\bf Period & \bf I & \bf II & \bf III & \bf IV & \bf V & \bf VI\\
\hline
MON & \multicolumn{2}{c|}{\cellcolor{orange}EW309 lab} & \multicolumn{2}{c|}{\cellcolor{orange}EW309 lab} & &\\
\hline
TUE & \multicolumn{6}{c|}{\cellcolor{yellow}Research / by appointment only}\\
\hline
WED & \multicolumn{2}{c|}{\cellcolor{orange}EW309 lab} & \multicolumn{2}{c|}{\cellcolor{orange}EW309 lab} & &\\
\hline
THU & & & & \multicolumn{2}{c|}{\cellcolor{orange}EW282C lab} & \\
\hline
FRI & & \multicolumn{2}{c|}{\cellcolor{orange}EW282D lab} & & & \\
\hline
\end{tabular}
\end{center}
\end{table}

\noclosing{}\\
%\signspace{}
\noindent\hspace*{4in}\includesignature{}
\signature{D Evangelista}
\sendertitle{Assistant Professor}

\noindent\hspace*{4in}{207 Maury Hall}\\
\hspace*{4in}{(410) 293-6132}\\
\hspace*{4in}{\href{mailto:evangeli@usna.edu}{evangeli@usna.edu}}

%CAPT T. A. Severson
%
%Office 	MU208, (410) 293-6111
%Email:	severson@usna.edu


\copyto{}
File\\
Course coordinator


% record note
\navyrecordnote
\thispagestyle{empty}

\navyrecordnotedistribution{%
Evangelista\\%
Devries\\
Kutzer\\
Severson}%
%\navyrecordnoteconcurrences{%
%\navyrecordnoteconcurrence{08K}
%\navyrecordnoteconcurrence{08I}}

\navyrecordnotesubjline

\section{}  This memo provides course policies for Evangelista's \usnaCourseTerm\ \usnaCourseNumber\ section.

\section{}  The policies here are adapted from those provided by CAPT Severson for EW309.

%\section{} File information: \gitMark
\end{document}


