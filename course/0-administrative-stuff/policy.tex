\documentclass[11pt,courier]{navymemo}

%\newcommand{\myroot}{../..}
%\usepackage{\myroot/ew305}
%incompatible for some reason?

\newcommand{\usnaCourseNumber}{EW305}
\newcommand{\usnaCourseName}{Linear Control Systems}
\newcommand{\usnaInstructor}{Assistant Professor D Evangelista}
\newcommand{\usnaInstructorShort}{D Evangelista}
\newcommand{\usnaCourseSection}{3321}
\newcommand{\usnaCourseTerm}{Fall 2019}

\author{\usnaInstructorShort}
\title{\usnaCourseNumber\ Course Policy}
\navysubj{Course policy for \usnaCourseNumber\ (\usnaCourseName)}
\navyfiling{1531}
%\navyserial{16-0001}
\date{\today}
%\navymarking{UNCLASSIFIED}

\usepackage{designature}
%\usepackage{siunitx}
\usepackage[colorlinks=true,urlcolor=blue]{hyperref}
%\usepackage{gitinfo2}

\usepackage{multirow}
\usepackage{xcolor,colortbl}
\usepackage{hhline}

\begin{document}
\makedateblock{}

\MEMORANDUM{}

\begin{navyletterheader}
\navyfrom{\usnaInstructor}
\navyto{\usnaCourseNumber\ Section \usnaCourseSection}
\navyskip{}%

\navysubjline{}
\navyskip{}%
\navyref{refa}{ACDEANINST 1531.58 (Administration of Academic Programs)}
\navyref{refb}{WEAPS\&SYSENGRINST 5400.16 (Teaching Methods \& Practices)}
\navyref{refc}{USNAINST 1531.53B (Policies Concerning Graded Academic Work)}
\navyref{refd}{SECNAVINSTA M-5216.5 (Department of Navy Correspondence Manual)}
\navyref{refe}{USNAINST 1610.3J (Birgade Honor Program)}
\navyref{reff}{COMDTMIDNINST 1752.1E (Sexual Assault Prevention and Response Program)}
%\navyskip{}%
\navyencl{encl1}{Syllabus, \usnaCourseNumber\ (\usnaCourseName), \usnaCourseTerm}
\end{navyletterheader}

\section{}
References (\ref{refa}) through (\ref{reff}) set forth my course policy for the \usnaCourseTerm\ session of \usnaCourseNumber\ (\usnaCourseName).  This information supplements the basic guidance provided in references~(\ref{refa}) through (\ref{reff}).

\section{Course Objectives and Learning Outcomes}  \usnaCourseNumber\ is an introduction to classical control systems, which comprises the mathematical modeling, time and frequency response analysos, and design of PID compensators.  Major topics include modeling physical systems with equations of motion, analysis of first and second order systems, root locus, and compensator and gain design. Material is supported by a series of laboratory projects that incorporate classroom concepts to design and implement control algorithms on physical systems. At the completion of this course, students must be able to:
\subsection{} Apply linear control design to meet system performance specifications
\subsection{} Understand the limits of linear control system performance
\subsection{} Implement continuous-time control design on a digital microcontroller (Tustin’s transform)
\subsection{} Understand a root locus plot for control design 
\subsection{} Understand the type number of a linear system and its relation to steady-state error
\subsection{} Analyze the response of a linear system to a step input
\subsection{} Identify interconnected subsystems (transfer functions) using a block diagram
\subsection{} Classify the mathematical model of a mechanical, electrical, or electro-mechanical system from its experimental response
\subsection{} Understand the process of parameter estimation for linear systems (first and second order)
\subsection{} Apply the Laplace Transform to represent linear differential equations in the frequency domain
%\subsection{} Model, identify, and verify a mathematical model of simple mechanical,electrical, and electromechanical systems.
%\subsection{} Analyze and predict the response of a linear system to a step input.
%\subsection{} Design and implement a linear control system to meet given specifications on the step response. 

\section{Required Textbook} Nise, N. S., (2015). \emph{Control Systems Engineering.} 7th ed. Wiley \& Sons.  Students are expected to complete reading assignments prior to class and to bring the text to class.

\section{Expectations}
\subsection{} Per USNA policy, eating and sleeping are prohibited during class. Use of phones and internet unrelated to class is also prohibited.  Drinks are permitted in closed containers.  Unless otherwise instructed, laptops will remain closed during lectures.  Professionalism befitting MIdshipmen of the United States Naval Academy is expected at all times. 
\subsection{Collaboration} Your shipmates are a valuable resource for understanding difficult concepts. Midshipmen are encouraged to work together on homework, as long as submitted work is completed individually. Use of outside sources/peers should be cited. Collaboration on quizzes or exams is strictly prohibited. \textbf{Plagiarism will be defined as any work, to include computer code, that is represented in any manner as your own original and unique work.}
\subsection{Late work} \textbf{Assignments are due by 2359 hours} on the date noted on the assignment.  Work turned in after the due date is subject to a 10\% penalty each day it is late. Assignments will not be accepted after solutions have been posted. Hardcopy is preferred.  
\subsection{Absences} If you will be absent for any reason, it is your responsibility to notify the instructor in advance and to obtain any class notes and assignments. It is your responsibility to submit assignments on time, even if you are absent. 

\section{Grading} The overall course grade will be approximately determined as listed. In addition to these criteria, preparation, participation, alertness class, and instructor's judgement of your progress throughout the course will be considered. \textbf{You must earn a passing grade in each grading area to earn a passing grade in this course.}
\subsection{} Homework assignments: 5\%
\subsection{} Hour exams (3): 10\% each
\subsection{} Lab assignments: 20\%
\subsection{} Lab quizzes and participation: 5\%
\subsection{} Final exam: 25\%
\subsection{} Final design project: 10\%

\section{Course Materials} Whenever possible, course materials will be posted on a shared Google drive. 

\section{Section Leader} The section leader will call the class to attention at the start and end of each class. If I have not arrived ten minutes after the start of class, the section leader should report to my office, then to the department office for instructions. If there are no instructions, the remaining class time should be used for work and study. 

\section{Coursework}
\subsection{} This class covers a large amount of material. Therefore, lectures and lab may not cover all the material covered on exams. Every effort will be made to discuss and identify the most important material.
\subsection{} The design project will evaluate your understanding of course material, and will be completed over several weeks.
\subsection{} All submitted work should be clear, concise, and of engineering quality. Submitted work should include:
\begin{enumerate}
\item Concise, clear description of the problem and steps taken toward your solution
\item Appropriate use of symbolic and/or numeric equations and solutions
\item Boxed or clearly indicated solution
\item You may use an FE-permitted calculator and notes/references as permitted. Calculators may not be shared during exams or quizzes. Smartphones may not be used in lieu of or in addition to calculators.  
\item Provide sufficient documentation to validate your results; this includes Simulink models, source code, input and parameter data, and results.
\item All plots must provide properly labeled axes, units, and should include a descriptive caption.  
\end{enumerate}

\clearpage
\section{Office Hours and Extra Instruction (EI)} Midshipmen are encouraged to arrange EI as needed (appointments encouraged). I am teaching EW485 (Comparative Biomechanics) MWF2, EW305 MWF3; EW305 lab T3-4; and EW281C (School of Drones) T5-6. Thursday is typically used for research but is available by appointment. There may be a Labrador retriever guide dog puppy in training at my office; if this is an issue for you I can make alternate arrangements. 
\begin{table}[h]
\begin{center}
\begin{tabular}[h]{| c | c | c | c | c | c | c |}
\hline
\bf Period & \bf I & \bf II & \bf III & \bf IV & \bf V & \bf VI\\
\hhline{~|*6-}
MON. & & \cellcolor{green}EW485E & \cellcolor{green}EW305-3433 & & &\\
\hhline{~|*6-}
TUE. & & & \multicolumn{2}{c}{\cellcolor{orange}\parbox[t]{3cm}{EW305 lab}} & \multicolumn{2}{c}{\cellcolor{orange}\parbox[t]{3cm}{EW281C}} \\
\hhline{~|*6-}
WED. & & \cellcolor{green}EW485E & \cellcolor{green}EW305-3433 & & &\\
\hhline{~|*6-}
THU. & \multicolumn{6}{c}{\cellcolor{yellow}Research / by appointment only}\\
\hhline{~|*6-}
FRI. & & \cellcolor{green}EW485E & \cellcolor{green}EW305-3433 & & &\\
\hline
\end{tabular}
\end{center}
\end{table}

\noclosing{}\\
%\signspace{}
\noindent\hspace*{4in}\includesignature{}
\signature{D Evangelista}
\sendertitle{Assistant Professor}

\noindent\hspace*{4in}{207 Maury Hall}\\
\hspace*{4in}{(410) 293-6132}\\
\hspace*{4in}{\href{mailto:evangeli@usna.edu}{evangeli@usna.edu}}

\copyto{}
File\\
Course coordinator


% record note
\navyrecordnote
\thispagestyle{empty}

\navyrecordnotedistribution{%
Evangelista\\%
Devries\\
Kiriakidis\\
O'Brien}%
%\navyrecordnoteconcurrences{%
%\navyrecordnoteconcurrence{08K}
%\navyrecordnoteconcurrence{08I}}

\navyrecordnotesubjline

\section{}  This memo provides course policies for Evangelista's \usnaCourseTerm\ \usnaCourseNumber\ section.

\section{}  The policies here are adapted from those provided by Devries for EW202 and EW305 using Navy standard memorandum format and to insert the updated 2020 EW305 learning outcomes.

%\section{} File information: \gitMark
\end{document}


