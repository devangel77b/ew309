\documentclass[aspectratio=169]{beamer}
\usepackage[utf8]{inputenc}
\usepackage[scaled]{helvet}
\renewcommand\familydefault{\sfdefault}
\usepackage[T1]{fontenc}

\newcommand{\myroot}{../..}
\usepackage[slides]{\myroot/course}
\title{Functional block diagrams}
\subtitle{\usnaCourseNumber\ Guided Design Experience, \usnaCourseTerm}
\author{\usnaInstructorShort}
\date{\today}
	
\usetheme{Hopper}

\begin{document}
\settitlebg
\begin{frame}
\titlepage
\end{frame}

\setslidebg
\begin{frame}
\frametitle{Purpose and types of technical documentation}
\begin{columns}
\begin{column}{0.5\textwidth}
Purpose
\begin{itemize}
\item Gives the customer and engineer a big picture view of a system
\item Breaks the larger system into smaller components to be designed or selected
\item Facilitates the division of labor among the engineering team
\end{itemize}
\end{column}
\begin{column}{0.5\textwidth}
Types
\begin{itemize}
\item Functional block diagrams
\item Interface control documents
\item Performance specifications
\item Pseudocode or algorithm
\item Technical reports
\end{itemize}
\end{column}
\end{columns}
\end{frame}

\begin{frame}
\frametitle{The functional block diagram}
\begin{itemize}
\item A functional block diagram is a \textbf{visual representation} of the subsystems and components showing the interconnections between each component and subsystem
\item Each block is a subsystem, with a label that includes the name and the component list
\item Each block is connected to other blocks 
\begin{itemize}
\item Each line matches one of three connection line styles:  data, electrical or mechanical.
\item \textbf{Each line is labeled with the specifics of the connection}
\item Lines should be laid out in an easy-to-read manner
\end{itemize}
\item A FBD should be \textbf{readable}, and include a legend and title
\end{itemize}
\end{frame}




\begin{frame}
\frametitle{EW202 Elevator Lab}
Graphic here
\end{frame}

\begin{frame}
\frametitle{FBD example: EW202 Elevator Lab}
Shitty graphic here
\end{frame}





\begin{frame}
\frametitle{Functional block diagrams help you}
\begin{itemize}
\item Communicate system architecture
\item Break a system down into manageable subsystems
\item Understand the interfaces between subsystems
\end{itemize}
\end{frame}

\begin{frame}
\frametitle{Content of the blocks}
\begin{itemize}
\item Each block should be described in detail
\item All connections should be shown
\item Any information transfer, protocols, voltages, etc. should be shown
\item Direction of flow should be annotated
\item Blocks may contain additional information:
\begin{itemize}
\item Pseudocode
\item Equation
\item Methodology
\item Citation if you will use someone else's method
\end{itemize}
\end{itemize}
\end{frame}

\begin{frame}
\frametitle{There are several types of connections on your FBD}
\begin{itemize}
\item Functional
\item Mechanical
\item Electrical connections
\item Data connections
\item Data protocols
\item Software function connectivity
\item All are important, but some are more important based on your project type.
\end{itemize}
\end{frame}

\begin{frame}
\frametitle{What type of connections do we have for the \usnaCourseNumber\ Autonomous Turret System?}
Ugly graphic here
\end{frame}

\begin{frame}
\frametitle{FBD Example: Autonomous Turret (Data commands/electrical)}
Ugly graphic here
\end{frame}

\begin{frame}
\frametitle{FBD Example: Autonomous Turret (Physical and data connections)}
Ugly graphic here
\end{frame}

% This is ridiculous. Do we need a 3 level list here?!?!
\begin{frame}
\frametitle{To do}
\begin{itemize}
\item Start working on a functional block diagram for your auto-turret (we will add to this as the semester goes on). 
\item What subsystems does your project currently have? 
\begin{itemize}
\item These should be represented as blocks on your FBD
\end{itemize}
\item What components belong to each of these subsystems?
\begin{itemize}
\item Components should be listed within their respective subsystem blocks
\item Components can also be their own blocks as well if they communicate directly with other components
\end{itemize}
\item For each subsystem and component, ask
\begin{itemize}
\item What are the inputs?
\item What are the outputs?
\item What type of connections are required 
\begin{itemize}
\item Electrical
\item Mechanical
\item Data
\item Other?
\end{itemize}
\end{itemize}
\item Don’t forget your power sources!
\end{itemize}
\end{frame}

\end{document}
